% Document type
\documentclass[10pt]{article}

% Custom commands
\newcommand{\todo}[1]{\textcolor{red}{\textbf{[TODO: #1}]}}
\newcommand{\e}[1]{\cdot 10^{#1}}
\newcommand{\hugeskip}{\bigskip\bigskip\bigskip}
\newcommand{\dott}{$\cdots\cdots$}
\newcommand{\del}{\bm{\nabla}}
\newcommand{\laplacian}{\nabla^2}
\newcommand{\cross}{\times}
\newcommand{\dd}[1]{\frac{d}{d{#1}}}
\newcommand{\tdd}[1]{\tfrac{d}{d{#1}}}
\newcommand{\pp}[1]{\frac{\partial}{\partial{#1}}}
\newcommand{\tpp}[1]{\tfrac{\partial}{\partial{#1}}}
\newcommand{\arccsc}{\text{arccsc}}
\newcommand{\arcsec}{\text{arcsec}}
\newcommand{\arccot}{\text{arccot}}
\newcommand{\longdel}{\biggl< \frac{\partial}{\partial x}, \frac{\partial}{\partial y}, \frac{\partial}{\partial z} \biggr>}
\newcommand{\si}{\text{s}}
\newcommand{\co}{\text{c}}
\newcommand{\unit}[1]{\hspace{.2em}\bm{\hat{#1}}}
\newcommand{\slantparallel}{\mathbin{\!/\mkern-5mu/\!}}
\newcommand{\mathdeg}{^\circ}

\renewcommand{\thefootnote}{\roman{footnote}} % Make footnotes numbered

\newcommand{\dataselectiontime}{$201$ hours ($254,838$ spacecraft spins) }
\newcommand{\backscattertime}{$26$ hours ($33,366$ spacecraft spins) }

% Package includes
\usepackage[utf8]{inputenc}
\usepackage{amssymb,amsmath,mathtools,amsthm,amscd,mathrsfs,graphicx,color} % Basics
\usepackage[cmtip,all,matrix,arrow,tips,curve]{xy} % For drawing diagrams
\usepackage[active]{srcltx} % Allows jumping from code <-> pdf
\usepackage{mathpazo} % Changes text font
\usepackage{newtxmath} % Changes math font
\usepackage{anyfontsize} % Fixes issues with font size
\usepackage{setspace}\onehalfspacing % Sets line spacing
\usepackage[skip=1em]{parskip} % Sets paragraph spacing
\usepackage[usenames,dvipsnames]{xcolor} % Provides color names
\usepackage[hyphens]{url} % Allows \url{} command for hyperlinks
\usepackage[bottom]{footmisc} % Forces footnotes to bottom of the page
\usepackage{cancel} % Provides cancellation symbol in equations
\usepackage{verbatim} % Provides comment environment and printing of literal text
\usepackage{bm} % For bold symbols
\usepackage{wrapfig} % Allows text to wrap around figures
\usepackage[labelfont=bf, justification=centering]{caption} % Makes figure labels bold and captions centered
\usepackage{float} % Allows forcing figure positions
\usepackage{tikz} % Allows creation of native figures
\usepackage{hhline} % Provides double hlines for tables
\usepackage{enumitem} % Allows for [nosep] and [noitemsep] in enumerate environment

% Colors
\definecolor{dark}{rgb}{0.77, 0.73, 0.675}

% Set up margins
\usepackage[headheight=65pt]{geometry}
\geometry{
  top = 1 in,
  inner = 1 in,
  outer = 1 in,
  bottom = 1 in,
  headheight = 3ex,
  headsep = 4ex,
}

% Set up header
\usepackage{fancyhdr}
\pagestyle{fancy}
\lhead{Julia Claxton 2025}
\chead{Supporting Information}
\rhead{}
\renewcommand{\headrulewidth}{1pt}

% Set up hyperlinks
\usepackage{hyperref}
\hypersetup{
  colorlinks,
  citecolor = black,
  filecolor = black,
  linkcolor = black,
  urlcolor = blue
}

% Box parameters
\setlength\fboxsep{.23cm}
\setlength\fboxrule{.02cm}

% Make sections un-numbered and no paragraph indents
\setcounter{secnumdepth}{0}
\setlength{\parindent}{0em}
%\setlength{\mathindent}{2em} % only if \documentclass[fleqn]{article} is used

% No separator line above footnotes
\renewcommand{\footnoterule}{}

% Show/hide labels
\usepackage[nolabel]{showlabels}
  % right
  % left
  % nolabel

% Bibliography options
\usepackage[bibstyle = nature]{biblatex} % Imports biblatex package
  % authortitle
  % draft
\addbibresource{citations.bib} % Import the bibliography file

% Make equations labelled as 1.1, 1.2, 2.3, etc.
\numberwithin{equation}{section}

% Set up title
\title{\doctitle}
\author{\me}
\date{}

\begin{document}
In this document we present a power law fit for backscatter ratio as a function of loss cone filling derived from ELFIN data. The data we are fitting to is shown in Figure \ref{figure:backscatter-vs-precipitation} (reproduced from manuscript). For this fit, we will only consider the case with low precipitating fluxes excluded, shown in the right panel.

\begin{figure}[H]
  \centering
  \noindent\includegraphics[width=0.9\textwidth]{figures/backscatter_vs_precipitation_ratio.png}
  \caption{
    Number backscatter ratio vs. loss cone filling strength derived from \backscattertime of ELFIN data. Any measurement of backscatter ratio with absolute uncertainty greater than $\sigma \approx 0.025$ was discarded. Left: Results all eligible ELFIN data (see manuscript). Right: Measurements with precipitating flux less than $10^{3.5}$~electrons-cm$^{-2}$-s$^{-1}$ removed. Energy backscatter ratio has a similar pattern to number backscatter ratio and is not shown for brevity.
  }
  \label{figure:backscatter-vs-precipitation}
\end{figure}

First we find the median backscatter ratio for every $J_\text{precip}/J_\text{trap}$ bin. This series, along with the $25$th and $75$th percentile of the $r_N$ distribution for each $J_\text{precip}/J_\text{trap}$ bin, is shown in Figure \ref{figure:curves} below.

\begin{figure}[H]
  \centering
  \noindent\includegraphics[width=0.9\textwidth]{figures/supplemental_curves.png}
  \caption{
    Median backscatter (black line) as well as the $25$th and $75$ percentile (grey ribbon below and above, respectively) as a function of precipitation strength. Left panel shows the number backscatter ratio $r_N$, right panel shows the energy backscatter $r_E$.
  }
  \label{figure:curves}
\end{figure}

We then fit a power law to each curve using a least-squares regression. The formulae for these fits and their coefficient of determination $R^2$ are given below.

\begin{align*}
  r_N &= 0.049846675 \cdot \left( \frac{J_\text{precip}}{J_\text{trap}} \right)^{-0.69318138} & R^2 = 0.9764 \\
  r_E &= 0.042889591 \cdot \left( \frac{J_\text{precip}}{J_\text{trap}} \right)^{-0.76185648} & R^2 =  0.9842
\end{align*}

These fits are plotted over the data in Figure \ref{figure:curvefit} below.

\begin{figure}[H]
  \centering
  \noindent\includegraphics[width=0.9\textwidth]{figures/supplemental_curvefit.png}
  \caption{
    Median backscatter rates (grey line with dots) at a variety of loss cone filling levels with a power law curvefit (black line) overlaid. Left: Number backscatter ratio as a function of loss cone filling. Right: Energy backscatter ratio as a fucntion of loss cone filling.
  }
  \label{figure:curvefit}
\end{figure}


\end{document}